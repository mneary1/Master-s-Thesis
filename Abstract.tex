%Abstract Page 

\hbox{\ }

\renewcommand{\baselinestretch}{1}
\small \normalsize

\begin{center}
\large{{ABSTRACT}} 

\vspace{3em} 

\end{center}
\hspace{-.15in}
\begin{tabular}{ll}
Title of thesis:    & {\large Identifying Extraneous Elements of Novice Source Code}
\ \\
&                          {\large Michael Patrick Neary III, Master of Science, 2018} \\
\ \\
Thesis directed by: & {\large  Professor Marie desJardins} \\
&  				{\large	 Department of Computer Science and Electrical Engineering} \\
\end{tabular}

\vspace{3em}

\renewcommand{\baselinestretch}{2}
\large \normalsize

Intelligent Tutoring Systems (ITS) are intelligent software systems that provide one-on-one guidance in a specific subject area. Introductory programming courses can use an ITS to give extra help to students learning to code for the first time. These systems generate constructive feedback to help the student understand the mistakes they make. Feedback is generated by analyzing correct programming solutions, looking for similarities with the student's code, to give a hint based on those similarities or lack thereof. The student's program might also have important information within its individual lines of code, as some such lines may be extraneous in the context of the problem. Current systems are not capable of meaningfully identifying these lines, or reasoning about it to give constructive feedback. I introduce a method for identification of extraneous lines of code, and show correct identification of extraneous lines in novice code artifacts.

