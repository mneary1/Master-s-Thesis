%!TEX root = mainthesis.tex

\renewcommand{\thechapter}{3}
\chapter{Related Work}
In this chapter, I provide related work in the generation of constructive feedback for Intelligent Programming Tutors. I discuss the strengths and drawbacks of each method.

\section{Constructive Feedback}

Quality feedback is important to facilitate the acquisition of new knowledge. The majority of feedback that a new programmer receives comes from the compiler or interpreter of the language they are using, which can be hard to decipher. This type of feedback is useless to the novice beyond highlighting syntax errors. It is therefore up to the instructor to provide meaningful feedback on student programming work, ranging from proper syntax use to walking through flawed logic. An IPT removes the need for the instructor to intervene and correct a flaw in student thinking. It should provide meaningful hints towards valid solutions and correct explanations of errors that arise when programming.

Stamper et al. laid the groundwork for constructive feedback in an intelligent tutor with the Hint Factory \cite{Stamper2013}. They created a system for a logic tutor that could generate step-by-step hints as a student was attempting to solve a problem. This system used data from past student problem solving attempts to create a Markov Decision Process, effectively creating a student model that was not individualized. Using this model, the system would figure out what the next best problem solving state was, and generate a hint to get the student into that state. Through the use of this hint generation method, Stamper et al. found that students increased the number of times they attempted to answer questions, and increased the number of questions they solved overall. The students who used the hint generation system achieved higher scores on post-tests in their experiments and received higher course grades overall. The Hint Factory was meant to be generalizable to all sorts of intelligent tutors, but there are limitations to using this system in a tutoring environment for programming.
%TODO: GO BACK TO THE HINT FACTORY PAPER AND NAME THE LIMITATIONS THAT IT PROPOSES

Generating hints for a student trying to solve a programming problem can be difficult. The solution space for a single programming problem can be infinitely large unless a bound is placed on the syntax or standard library functions that are allowed in a solution. Searching through a solution space to find the most similar solution to what a student has is therefore impractical. Current methods for hint generation in this space either circumvent the need for looking at the solution space, or take steps to pare down its size.

There have been a few approaches for improving hint generation and programming feedback in an IPT. Recently, there has been a push towards the use of programming data to generate feedback. One method was developed by Lazar and Bratko without the need for a state-space representation of the problem solving process, instead analyzing student programs by looking for common “edits” between them \cite{Lazar}. Once these edits were found, the authors applied them to a student submission until a proper solution was reached. From the sequence of applied edits, they derived hints they would show to the student. Using this approach, they were able to fix 70\% of student submissions, independent of language choice, without the need to manually enter common mistakes for the system to be aware of.

Other approaches treat the problem solving environment of learning to program with a state-space representation method. Koedinger et al. generate a solution space using Abstract Syntax Trees created from correct student code samples, and they reduce the size of this solution space by applying certain transformations on these ASTs \cite{Koedinger2013}. Using the reduced solution space, they create the AST for a student's intermediate submission, and then compute the difference between the intermediate solution and the known solutions. The goal is then to reverse the differences to generate feedback from the closest solution found.

Expanding on their previous results, Rivers and Koedinger describe a path construction method, where instead of comparing all possible solutions, they develop a space of possible paths that a student may take in order to get to a solution, and generate hints based on the most likely path that a student could take, given the previous attempts \cite{Rivers2016}. This method was more computationally feasible than generating the solution space, and they found most hints to be generated fairly quickly (in less than one minute). However, they are not certain how helpful the hints generated were for the students interacting with the system.
