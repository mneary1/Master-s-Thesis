\appendix
\renewcommand{\thechapter}{A}
\renewcommand{\chaptername}{Appendix}
\chapter{Problem 1 Data}
This appendix is referenced in Chapter 5 in the analysis of my method applied to the data I have for the first problem. A positive result in this context means that a line was identified as extraneous, and a negative results means that a line was not identified as extraneous. A confusion matrix, accuracy, sensitivity, specificity, and precision were computed for each trial. 

\begin{table}
\caption{Intersection of Both Label Sets with All Three Dependencies}
\begin{minipage}{.6\textwidth}
\centering
\begin{tabular}{l|ll}
\backslashbox{Results}{Actual} & Positive & Negative \\ \hline
Positive & 4 & 372 \\
Negative & 72 & 8031 \\
\end{tabular}
\subcaption{Confusion Matrix}
\end{minipage}
\begin{minipage}{.6\textwidth}
\centering
\begin{tabular}{l|ll}
Accuracy & 0.9476 \\ \hline
Sensitivity & 0.0526 \\ \hline
Specificity & 0.9557 \\ \hline
Precision & 0.0106 \\
\end{tabular}
\subcaption{Statistics}
\end{minipage}
\end{table}

\begin{table}
\caption{Student A's Labels with All Three Dependencies}
\begin{minipage}{.6\textwidth}
\centering
\begin{tabular}{l|ll}
\backslashbox{Results}{Actual} & Positive & Negative \\ \hline
Positive & 8 & 368 \\
Negative & 130 & 7973 \\
\end{tabular}
\subcaption{Confusion Matrix}
\end{minipage}
\begin{minipage}{.6\textwidth}
\centering
\begin{tabular}{l|ll}
Accuracy & 0.9413 \\ \hline
Sensitivity & 0.058 \\ \hline
Specificity & 0.9559 \\ \hline
Precision & 0.0213 \\
\end{tabular}
\subcaption{Statistics}
\end{minipage}
\end{table}

\begin{table}
\caption{Student B's Labels with All Three Dependencies}
\begin{minipage}{.6\textwidth}
\centering
\begin{tabular}{l|ll}
\backslashbox{Results}{Actual} & Positive & Negative \\ \hline
Positive & 139 & 237 \\
Negative & 436 & 7667 \\
\end{tabular}
\subcaption{Confusion Matrix}
\end{minipage}
\begin{minipage}{.6\textwidth}
\centering
\begin{tabular}{l|ll}
Accuracy & 0.9206 \\ \hline
Sensitivity & 0.2417 \\ \hline
Specificity & 0.97 \\ \hline
Precision & 0.3697 \\
\end{tabular}
\subcaption{Statistics}
\end{minipage}
\end{table}

\begin{table}
\caption{Union of Both Label Sets with All Three Dependencies}
\begin{minipage}{.6\textwidth}
\centering
\begin{tabular}{l|ll}
\backslashbox{Results}{Actual} & Positive & Negative \\ \hline
Positive & 143 & 233 \\
Negative & 494 & 7609 \\
\end{tabular}
\subcaption{Confusion Matrix}
\end{minipage}
\begin{minipage}{.6\textwidth}
\centering
\begin{tabular}{l|ll}
Accuracy & 0.9143 \\ \hline
Sensitivity & 0.2245 \\ \hline
Specificity & 0.9703 \\ \hline
Precision & 0.3803 \\
\end{tabular}
\subcaption{Statistics}
\end{minipage}
\end{table}


\begin{table}
\caption{Intersection of Both Label Sets without Data Dependencies}
\begin{minipage}{.6\textwidth}
\centering
\begin{tabular}{l|ll}
\backslashbox{Results}{Actual} & Positive & Negative \\ \hline
Positive & 60 & 1780 \\
Negative & 16 & 6623 \\
\end{tabular}
\subcaption{Confusion Matrix}
\end{minipage}
\begin{minipage}{.6\textwidth}
\centering
\begin{tabular}{l|ll}
Accuracy & 0.7882 \\ \hline
Sensitivity & 0.7895 \\ \hline
Specificity & 0.7882 \\ \hline
Precision & 0.0326 \\
\end{tabular}
\subcaption{Statistics}
\end{minipage}
\end{table}

\begin{table}
\caption{Student A's Labels without Data Dependencies}
\begin{minipage}{.6\textwidth}
\centering
\begin{tabular}{l|ll}
\backslashbox{Results}{Actual} & Positive & Negative \\ \hline
Positive & 91 & 1749 \\
Negative & 47 & 6592 \\
\end{tabular}
\subcaption{Confusion Matrix}
\end{minipage}
\begin{minipage}{.6\textwidth}
\centering
\begin{tabular}{l|ll}
Accuracy & 0.7882 \\ \hline
Sensitivity & 0.6594 \\ \hline
Specificity & 0.7903 \\ \hline
Precision & 0.0495 \\
\end{tabular}
\subcaption{Statistics}
\end{minipage}
\end{table}

\begin{table}
\caption{Student B's Labels without Data Dependencies}
\begin{minipage}{.6\textwidth}
\centering
\begin{tabular}{l|ll}
\backslashbox{Results}{Actual} & Positive & Negative \\ \hline
Positive & 221 & 1619 \\
Negative & 354 & 6285 \\
\end{tabular}
\subcaption{Confusion Matrix}
\end{minipage}
\begin{minipage}{.6\textwidth}
\centering
\begin{tabular}{l|ll}
Accuracy & 0.7673 \\ \hline
Sensitivity & 0.3843 \\ \hline
Specificity & 0.7952 \\ \hline
Precision & 0.1201 \\
\end{tabular}
\subcaption{Statistics}
\end{minipage}
\end{table}

\begin{table}
\caption{Union of Both Label Sets without Data Dependencies}
\begin{minipage}{.6\textwidth}
\centering
\begin{tabular}{l|ll}
\backslashbox{Results}{Actual} & Positive & Negative \\ \hline
Positive & 252 & 1588 \\
Negative & 385 & 6254 \\
\end{tabular}
\subcaption{Confusion Matrix}
\end{minipage}
\begin{minipage}{.6\textwidth}
\centering
\begin{tabular}{l|ll}
Accuracy & 0.7673 \\ \hline
Sensitivity & 0.3956 \\ \hline
Specificity & 0.7975 \\ \hline
Precision & 0.137 \\
\end{tabular}
\subcaption{Statistics}
\end{minipage}
\end{table}


\begin{table}
\caption{Intersection of Both Label Sets without Execution Dependencies}
\begin{minipage}{.6\textwidth}
\centering
\begin{tabular}{l|ll}
\backslashbox{Results}{Actual} & Positive & Negative \\ \hline
Positive & 56 & 1765 \\
Negative & 20 & 6638 \\
\end{tabular}
\subcaption{Confusion Matrix}
\end{minipage}
\begin{minipage}{.6\textwidth}
\centering
\begin{tabular}{l|ll}
Accuracy & 0.7895 \\ \hline
Sensitivity & 0.7368 \\ \hline
Specificity & 0.79 \\ \hline
Precision & 0.0308 \\
\end{tabular}
\subcaption{Statistics}
\end{minipage}
\end{table}

\begin{table}
\caption{Student A's Labels without Execution Dependencies}
\begin{minipage}{.6\textwidth}
\centering
\begin{tabular}{l|ll}
\backslashbox{Results}{Actual} & Positive & Negative \\ \hline
Positive & 81 & 1740 \\
Negative & 57 & 6601 \\
\end{tabular}
\subcaption{Confusion Matrix}
\end{minipage}
\begin{minipage}{.6\textwidth}
\centering
\begin{tabular}{l|ll}
Accuracy & 0.7881 \\ \hline
Sensitivity & 0.587 \\ \hline
Specificity & 0.7914 \\ \hline
Precision & 0.0445 \\
\end{tabular}
\subcaption{Statistics}
\end{minipage}
\end{table}

\begin{table}
\caption{Student B's Labels without Execution Dependencies}
\begin{minipage}{.6\textwidth}
\centering
\begin{tabular}{l|ll}
\backslashbox{Results}{Actual} & Positive & Negative \\ \hline
Positive & 218 & 1603 \\
Negative & 357 & 6301 \\
\end{tabular}
\subcaption{Confusion Matrix}
\end{minipage}
\begin{minipage}{.6\textwidth}
\centering
\begin{tabular}{l|ll}
Accuracy & 0.7688 \\ \hline
Sensitivity & 0.3791 \\ \hline
Specificity & 0.7972 \\ \hline
Precision & 0.1197 \\
\end{tabular}
\subcaption{Statistics}
\end{minipage}
\end{table}

\begin{table}
\caption{Union of Both Label Sets without Execution Dependencies}
\begin{minipage}{.6\textwidth}
\centering
\begin{tabular}{l|ll}
\backslashbox{Results}{Actual} & Positive & Negative \\ \hline
Positive & 243 & 1578 \\
Negative & 394 & 6264 \\
\end{tabular}
\subcaption{Confusion Matrix}
\end{minipage}
\begin{minipage}{.6\textwidth}
\centering
\begin{tabular}{l|ll}
Accuracy & 0.7674 \\ \hline
Sensitivity & 0.3815 \\ \hline
Specificity & 0.7988 \\ \hline
Precision & 0.1334 \\
\end{tabular}
\subcaption{Statistics}
\end{minipage}
\end{table}


\begin{table}
\caption{Intersection of Both Label Sets without Structural Dependencies}
\begin{minipage}{.6\textwidth}
\centering
\begin{tabular}{l|ll}
\backslashbox{Results}{Actual} & Positive & Negative \\ \hline
Positive & 59 & 1843 \\
Negative & 17 & 6560 \\
\end{tabular}
\subcaption{Confusion Matrix}
\end{minipage}
\begin{minipage}{.6\textwidth}
\centering
\begin{tabular}{l|ll}
Accuracy & 0.7806 \\ \hline
Sensitivity & 0.7763 \\ \hline
Specificity & 0.7807 \\ \hline
Precision & 0.031 \\
\end{tabular}
\subcaption{Statistics}
\end{minipage}
\end{table}

\begin{table}
\caption{Student A's Labels without Structural Dependencies}
\begin{minipage}{.6\textwidth}
\centering
\begin{tabular}{l|ll}
\backslashbox{Results}{Actual} & Positive & Negative \\ \hline
Positive & 90 & 1812 \\
Negative & 48 & 6529 \\
\end{tabular}
\subcaption{Confusion Matrix}
\end{minipage}
\begin{minipage}{.6\textwidth}
\centering
\begin{tabular}{l|ll}
Accuracy & 0.7806 \\ \hline
Sensitivity & 0.6522 \\ \hline
Specificity & 0.7828 \\ \hline
Precision & 0.0473 \\
\end{tabular}
\subcaption{Statistics}
\end{minipage}
\end{table}

\begin{table}
\caption{Student B's Labels without Structural Dependencies}
\begin{minipage}{.6\textwidth}
\centering
\begin{tabular}{l|ll}
\backslashbox{Results}{Actual} & Positive & Negative \\ \hline
Positive & 220 & 1682 \\
Negative & 355 & 6222 \\
\end{tabular}
\subcaption{Confusion Matrix}
\end{minipage}
\begin{minipage}{.6\textwidth}
\centering
\begin{tabular}{l|ll}
Accuracy & 0.7598 \\ \hline
Sensitivity & 0.3826 \\ \hline
Specificity & 0.7872 \\ \hline
Precision & 0.1157 \\
\end{tabular}
\subcaption{Statistics}
\end{minipage}
\end{table}

\begin{table}
\caption{Union of Both Label Sets without Structural Dependencies}
\begin{minipage}{.6\textwidth}
\centering
\begin{tabular}{l|ll}
\backslashbox{Results}{Actual} & Positive & Negative \\ \hline
Positive & 251 & 1651 \\
Negative & 386 & 6191 \\
\end{tabular}
\subcaption{Confusion Matrix}
\end{minipage}
\begin{minipage}{.6\textwidth}
\centering
\begin{tabular}{l|ll}
Accuracy & 0.7598 \\ \hline
Sensitivity & 0.394 \\ \hline
Specificity & 0.7895 \\ \hline
Precision & 0.132 \\
\end{tabular}
\subcaption{Statistics}
\end{minipage}
\end{table}


\begin{table}
\caption{Intersection of Both Label Sets With Only Data Dependencies}
\begin{minipage}{.6\textwidth}
\centering
\begin{tabular}{l|ll}
\backslashbox{Results}{Actual} & Positive & Negative \\ \hline
Positive & 59 & 2190 \\
Negative & 17 & 6213 \\
\end{tabular}
\subcaption{Confusion Matrix}
\end{minipage}
\begin{minipage}{.6\textwidth}
\centering
\begin{tabular}{l|ll}
Accuracy & 0.7397 \\ \hline
Sensitivity & 0.7763 \\ \hline
Specificity & 0.7394 \\ \hline
Precision & 0.0262 \\
\end{tabular}
\subcaption{Statistics}
\end{minipage}
\end{table}

\begin{table}
\caption{Student A's Labels With Only Data Dependencies}
\begin{minipage}{.6\textwidth}
\centering
\begin{tabular}{l|ll}
\backslashbox{Results}{Actual} & Positive & Negative \\ \hline
Positive & 87 & 2162 \\
Negative & 51 & 6179 \\
\end{tabular}
\subcaption{Confusion Matrix}
\end{minipage}
\begin{minipage}{.6\textwidth}
\centering
\begin{tabular}{l|ll}
Accuracy & 0.739 \\ \hline
Sensitivity & 0.6304 \\ \hline
Specificity & 0.7408 \\ \hline
Precision & 0.0387 \\
\end{tabular}
\subcaption{Statistics}
\end{minipage}
\end{table}

\begin{table}
\caption{Student B's Labels With Only Data Dependencies}
\begin{minipage}{.6\textwidth}
\centering
\begin{tabular}{l|ll}
\backslashbox{Results}{Actual} & Positive & Negative \\ \hline
Positive & 234 & 2015 \\
Negative & 341 & 5889 \\
\end{tabular}
\subcaption{Confusion Matrix}
\end{minipage}
\begin{minipage}{.6\textwidth}
\centering
\begin{tabular}{l|ll}
Accuracy & 0.7221 \\ \hline
Sensitivity & 0.407 \\ \hline
Specificity & 0.7451 \\ \hline
Precision & 0.104 \\
\end{tabular}
\subcaption{Statistics}
\end{minipage}
\end{table}

\begin{table}
\caption{Union of Both Label Sets With Only Data Dependencies}
\begin{minipage}{.6\textwidth}
\centering
\begin{tabular}{l|ll}
\backslashbox{Results}{Actual} & Positive & Negative \\ \hline
Positive & 262 & 1987 \\
Negative & 375 & 5855 \\
\end{tabular}
\subcaption{Confusion Matrix}
\end{minipage}
\begin{minipage}{.6\textwidth}
\centering
\begin{tabular}{l|ll}
Accuracy & 0.7214 \\ \hline
Sensitivity & 0.4113 \\ \hline
Specificity & 0.7466 \\ \hline
Precision & 0.1165 \\
\end{tabular}
\subcaption{Statistics}
\end{minipage}
\end{table}


\begin{table}
\caption{Intersection of Both Label Sets With Only Execution Dependencies}
\begin{minipage}{.6\textwidth}
\centering
\begin{tabular}{l|ll}
\backslashbox{Results}{Actual} & Positive & Negative \\ \hline
Positive & 57 & 1886 \\
Negative & 19 & 6517 \\
\end{tabular}
\subcaption{Confusion Matrix}
\end{minipage}
\begin{minipage}{.6\textwidth}
\centering
\begin{tabular}{l|ll}
Accuracy & 0.7753 \\ \hline
Sensitivity & 0.75 \\ \hline
Specificity & 0.7756 \\ \hline
Precision & 0.0293 \\
\end{tabular}
\subcaption{Statistics}
\end{minipage}
\end{table}

\begin{table}
\caption{Student A's Labels With Only Execution Dependencies}
\begin{minipage}{.6\textwidth}
\centering
\begin{tabular}{l|ll}
\backslashbox{Results}{Actual} & Positive & Negative \\ \hline
Positive & 87 & 1856 \\
Negative & 51 & 6485 \\
\end{tabular}
\subcaption{Confusion Matrix}
\end{minipage}
\begin{minipage}{.6\textwidth}
\centering
\begin{tabular}{l|ll}
Accuracy & 0.7751 \\ \hline
Sensitivity & 0.6304 \\ \hline
Specificity & 0.7775 \\ \hline
Precision & 0.0448 \\
\end{tabular}
\subcaption{Statistics}
\end{minipage}
\end{table}

\begin{table}
\caption{Student B's Labels With Only Execution Dependencies}
\begin{minipage}{.6\textwidth}
\centering
\begin{tabular}{l|ll}
\backslashbox{Results}{Actual} & Positive & Negative \\ \hline
Positive & 215 & 1728 \\
Negative & 360 & 6176 \\
\end{tabular}
\subcaption{Confusion Matrix}
\end{minipage}
\begin{minipage}{.6\textwidth}
\centering
\begin{tabular}{l|ll}
Accuracy & 0.7537 \\ \hline
Sensitivity & 0.3739 \\ \hline
Specificity & 0.7814 \\ \hline
Precision & 0.1107 \\
\end{tabular}
\subcaption{Statistics}
\end{minipage}
\end{table}

\begin{table}
\caption{Union of Both Label Sets With Only Execution Dependencies}
\begin{minipage}{.6\textwidth}
\centering
\begin{tabular}{l|ll}
\backslashbox{Results}{Actual} & Positive & Negative \\ \hline
Positive & 245 & 1698 \\
Negative & 392 & 6144 \\
\end{tabular}
\subcaption{Confusion Matrix}
\end{minipage}
\begin{minipage}{.6\textwidth}
\centering
\begin{tabular}{l|ll}
Accuracy & 0.7535 \\ \hline
Sensitivity & 0.3846 \\ \hline
Specificity & 0.7835 \\ \hline
Precision & 0.1261 \\
\end{tabular}
\subcaption{Statistics}
\end{minipage}
\end{table}


\begin{table}
\caption{Intersection of Both Label Sets With Only Structural Dependencies}
\begin{minipage}{.6\textwidth}
\centering
\begin{tabular}{l|ll}
\backslashbox{Results}{Actual} & Positive & Negative \\ \hline
Positive & 59 & 1955 \\
Negative & 17 & 6448 \\
\end{tabular}
\subcaption{Confusion Matrix}
\end{minipage}
\begin{minipage}{.6\textwidth}
\centering
\begin{tabular}{l|ll}
Accuracy & 0.7674 \\ \hline
Sensitivity & 0.7763 \\ \hline
Specificity & 0.7673 \\ \hline
Precision & 0.0293 \\
\end{tabular}
\subcaption{Statistics}
\end{minipage}
\end{table}

\begin{table}
\caption{Student A's Labels With Only Structural Dependencies}
\begin{minipage}{.6\textwidth}
\centering
\begin{tabular}{l|ll}
\backslashbox{Results}{Actual} & Positive & Negative \\ \hline
Positive & 88 & 1926 \\
Negative & 50 & 6415 \\
\end{tabular}
\subcaption{Confusion Matrix}
\end{minipage}
\begin{minipage}{.6\textwidth}
\centering
\begin{tabular}{l|ll}
Accuracy & 0.767 \\ \hline
Sensitivity & 0.6377 \\ \hline
Specificity & 0.7691 \\ \hline
Precision & 0.0437 \\
\end{tabular}
\subcaption{Statistics}
\end{minipage}
\end{table}

\begin{table}
\caption{Student B's Labels With Only Structural Dependencies}
\begin{minipage}{.6\textwidth}
\centering
\begin{tabular}{l|ll}
\backslashbox{Results}{Actual} & Positive & Negative \\ \hline
Positive & 220 & 1794 \\
Negative & 355 & 6110 \\
\end{tabular}
\subcaption{Confusion Matrix}
\end{minipage}
\begin{minipage}{.6\textwidth}
\centering
\begin{tabular}{l|ll}
Accuracy & 0.7466 \\ \hline
Sensitivity & 0.3826 \\ \hline
Specificity & 0.773 \\ \hline
Precision & 0.1092 \\
\end{tabular}
\subcaption{Statistics}
\end{minipage}
\end{table}

\begin{table}
\caption{Union of Both Label Sets With Only Structural Dependencies}
\begin{minipage}{.6\textwidth}
\centering
\begin{tabular}{l|ll}
\backslashbox{Results}{Actual} & Positive & Negative \\ \hline
Positive & 249 & 1765 \\
Negative & 388 & 6077 \\
\end{tabular}
\subcaption{Confusion Matrix}
\end{minipage}
\begin{minipage}{.6\textwidth}
\centering
\begin{tabular}{l|ll}
Accuracy & 0.7461 \\ \hline
Sensitivity & 0.3909 \\ \hline
Specificity & 0.7749 \\ \hline
Precision & 0.1236 \\
\end{tabular}
\subcaption{Statistics}
\end{minipage}
\end{table}
