\appendix
\renewcommand{\thechapter}{B}
\renewcommand{\chaptername}Appendix}
\chapter{Problem 2 Data}

This appendix is referenced in Chapter 5 in the analysis of my method applied to the data I have for the second problem. A positive result in this context means that a line was identified as extraneous, and a negative results means that a line was not identified as extraneous. A confusion matrix, accuracy, sensitivity, specificity, and precision were computed for each trial. 

\begin{table}
\caption{Title Goes Here}
\begin{minipage}{.6\textwidth}
\centering
\begin{tabular}{l|ll}
\backslashbox{Results}{Actual} & Positive & Negative \\ \hline
Positive & 0 & 16 \\
Negative & 104 & 6196 \\
\end{tabular}
\subcaption{Confusion Matrix}
\end{minipage}
\begin{minipage}{.6\textwidth}
\centering
\begin{tabular}{l|ll}
Accuracy & 0.981 \\ \hline
Sensitivity & 0.0 \\ \hline
Specificity & 0.9974 \\ \hline
Precision & 0.0 \\
\end{tabular}
\subcaption{Statistics}
\end{minipage}
\end{table}

\begin{table}
\caption{Student A's Labels with All Three Dependencies}
\begin{minipage}{.6\textwidth}
\centering
\begin{tabular}{l|ll}
\backslashbox{Results}{Actual} & Positive & Negative \\ \hline
Positive & 8 & 368 \\
Negative & 130 & 7973 \\
\end{tabular}
\subcaption{Confusion Matrix}
\end{minipage}
\begin{minipage}{.6\textwidth}
\centering
\begin{tabular}{l|ll}
Accuracy & 0.9413 \\ \hline
Sensitivity & 0.058 \\ \hline
Specificity & 0.9559 \\ \hline
Precision & 0.0213 \\
\end{tabular}
\subcaption{Statistics}
\end{minipage}
\end{table}

\begin{table}
\caption{Title Goes Here}
\begin{minipage}{.6\textwidth}
\centering
\begin{tabular}{l|ll}
\backslashbox{Results}{Actual} & Positive & Negative \\ \hline
Positive & 0 & 16 \\
Negative & 199 & 6102 \\
\end{tabular}
\subcaption{Confusion Matrix}
\end{minipage}
\begin{minipage}{.6\textwidth}
\centering
\begin{tabular}{l|ll}
Accuracy & 0.966 \\ \hline
Sensitivity & 0.0 \\ \hline
Specificity & 0.9974 \\ \hline
Precision & 0.0 \\
\end{tabular}
\subcaption{Statistics}
\end{minipage}
\end{table}

\begin{table}
\caption{Title Goes Here}
\begin{minipage}{.6\textwidth}
\centering
\begin{tabular}{l|ll}
\backslashbox{Results}{Actual} & Positive & Negative \\ \hline
Positive & 0 & 16 \\
Negative & 435 & 5868 \\
\end{tabular}
\subcaption{Confusion Matrix}
\end{minipage}
\begin{minipage}{.6\textwidth}
\centering
\begin{tabular}{l|ll}
Accuracy & 0.9286 \\ \hline
Sensitivity & 0.0 \\ \hline
Specificity & 0.9973 \\ \hline
Precision & 0.0 \\
\end{tabular}
\subcaption{Statistics}
\end{minipage}
\end{table}


\begin{table}
\caption{Intersection of Both Label Sets without Data Dependencies}
\begin{minipage}{.6\textwidth}
\centering
\begin{tabular}{l|ll}
\backslashbox{Results}{Actual} & Positive & Negative \\ \hline
Positive & 60 & 1780 \\
Negative & 16 & 6623 \\
\end{tabular}
\subcaption{Confusion Matrix}
\end{minipage}
\begin{minipage}{.6\textwidth}
\centering
\begin{tabular}{l|ll}
Accuracy & 0.7882 \\ \hline
Sensitivity & 0.7895 \\ \hline
Specificity & 0.7882 \\ \hline
Precision & 0.0326 \\
\end{tabular}
\subcaption{Statistics}
\end{minipage}
\end{table}

\begin{table}
\caption{Student A's Labels without Data Dependencies}
\begin{minipage}{.6\textwidth}
\centering
\begin{tabular}{l|ll}
\backslashbox{Results}{Actual} & Positive & Negative \\ \hline
Positive & 91 & 1749 \\
Negative & 47 & 6592 \\
\end{tabular}
\subcaption{Confusion Matrix}
\end{minipage}
\begin{minipage}{.6\textwidth}
\centering
\begin{tabular}{l|ll}
Accuracy & 0.7882 \\ \hline
Sensitivity & 0.6594 \\ \hline
Specificity & 0.7903 \\ \hline
Precision & 0.0495 \\
\end{tabular}
\subcaption{Statistics}
\end{minipage}
\end{table}

\begin{table}
\caption{Title Goes Here}
\begin{minipage}{.6\textwidth}
\centering
\begin{tabular}{l|ll}
\backslashbox{Results}{Actual} & Positive & Negative \\ \hline
Positive & 2 & 36 \\
Negative & 197 & 6082 \\
\end{tabular}
\subcaption{Confusion Matrix}
\end{minipage}
\begin{minipage}{.6\textwidth}
\centering
\begin{tabular}{l|ll}
Accuracy & 0.9631 \\ \hline
Sensitivity & 0.0101 \\ \hline
Specificity & 0.9941 \\ \hline
Precision & 0.0526 \\
\end{tabular}
\subcaption{Statistics}
\end{minipage}
\end{table}

\begin{table}
\caption{Title Goes Here}
\begin{minipage}{.6\textwidth}
\centering
\begin{tabular}{l|ll}
\backslashbox{Results}{Actual} & Positive & Negative \\ \hline
Positive & 2 & 36 \\
Negative & 433 & 5848 \\
\end{tabular}
\subcaption{Confusion Matrix}
\end{minipage}
\begin{minipage}{.6\textwidth}
\centering
\begin{tabular}{l|ll}
Accuracy & 0.9258 \\ \hline
Sensitivity & 0.0046 \\ \hline
Specificity & 0.9939 \\ \hline
Precision & 0.0526 \\
\end{tabular}
\subcaption{Statistics}
\end{minipage}
\end{table}


\begin{table}
\caption{Intersection of Both Label Sets without Execution Dependencies}
\begin{minipage}{.6\textwidth}
\centering
\begin{tabular}{l|ll}
\backslashbox{Results}{Actual} & Positive & Negative \\ \hline
Positive & 56 & 1765 \\
Negative & 20 & 6638 \\
\end{tabular}
\subcaption{Confusion Matrix}
\end{minipage}
\begin{minipage}{.6\textwidth}
\centering
\begin{tabular}{l|ll}
Accuracy & 0.7895 \\ \hline
Sensitivity & 0.7368 \\ \hline
Specificity & 0.79 \\ \hline
Precision & 0.0308 \\
\end{tabular}
\subcaption{Statistics}
\end{minipage}
\end{table}

\begin{table}
\caption{Title Goes Here}
\begin{minipage}{.6\textwidth}
\centering
\begin{tabular}{l|ll}
\backslashbox{Results}{Actual} & Positive & Negative \\ \hline
Positive & 0 & 42 \\
Negative & 340 & 5936 \\
\end{tabular}
\subcaption{Confusion Matrix}
\end{minipage}
\begin{minipage}{.6\textwidth}
\centering
\begin{tabular}{l|ll}
Accuracy & 0.9395 \\ \hline
Sensitivity & 0.0 \\ \hline
Specificity & 0.993 \\ \hline
Precision & 0.0 \\
\end{tabular}
\subcaption{Statistics}
\end{minipage}
\end{table}

\begin{table}
\caption{Title Goes Here}
\begin{minipage}{.6\textwidth}
\centering
\begin{tabular}{l|ll}
\backslashbox{Results}{Actual} & Positive & Negative \\ \hline
Positive & 0 & 42 \\
Negative & 199 & 6076 \\
\end{tabular}
\subcaption{Confusion Matrix}
\end{minipage}
\begin{minipage}{.6\textwidth}
\centering
\begin{tabular}{l|ll}
Accuracy & 0.9618 \\ \hline
Sensitivity & 0.0 \\ \hline
Specificity & 0.9931 \\ \hline
Precision & 0.0 \\
\end{tabular}
\subcaption{Statistics}
\end{minipage}
\end{table}

\begin{table}
\caption{Union of Both Label Sets without Execution Dependencies}
\begin{minipage}{.6\textwidth}
\centering
\begin{tabular}{l|ll}
\backslashbox{Results}{Actual} & Positive & Negative \\ \hline
Positive & 243 & 1578 \\
Negative & 394 & 6264 \\
\end{tabular}
\subcaption{Confusion Matrix}
\end{minipage}
\begin{minipage}{.6\textwidth}
\centering
\begin{tabular}{l|ll}
Accuracy & 0.7674 \\ \hline
Sensitivity & 0.3815 \\ \hline
Specificity & 0.7988 \\ \hline
Precision & 0.1334 \\
\end{tabular}
\subcaption{Statistics}
\end{minipage}
\end{table}


\begin{table}
\caption{Title Goes Here}
\begin{minipage}{.6\textwidth}
\centering
\begin{tabular}{l|ll}
\backslashbox{Results}{Actual} & Positive & Negative \\ \hline
Positive & 2 & 34 \\
Negative & 102 & 6178 \\
\end{tabular}
\subcaption{Confusion Matrix}
\end{minipage}
\begin{minipage}{.6\textwidth}
\centering
\begin{tabular}{l|ll}
Accuracy & 0.9785 \\ \hline
Sensitivity & 0.0192 \\ \hline
Specificity & 0.9945 \\ \hline
Precision & 0.0556 \\
\end{tabular}
\subcaption{Statistics}
\end{minipage}
\end{table}

\begin{table}
\caption{Title Goes Here}
\begin{minipage}{.6\textwidth}
\centering
\begin{tabular}{l|ll}
\backslashbox{Results}{Actual} & Positive & Negative \\ \hline
Positive & 2 & 34 \\
Negative & 338 & 5944 \\
\end{tabular}
\subcaption{Confusion Matrix}
\end{minipage}
\begin{minipage}{.6\textwidth}
\centering
\begin{tabular}{l|ll}
Accuracy & 0.9411 \\ \hline
Sensitivity & 0.0059 \\ \hline
Specificity & 0.9943 \\ \hline
Precision & 0.0556 \\
\end{tabular}
\subcaption{Statistics}
\end{minipage}
\end{table}

\begin{table}
\caption{Title Goes Here}
\begin{minipage}{.6\textwidth}
\centering
\begin{tabular}{l|ll}
\backslashbox{Results}{Actual} & Positive & Negative \\ \hline
Positive & 2 & 34 \\
Negative & 197 & 6084 \\
\end{tabular}
\subcaption{Confusion Matrix}
\end{minipage}
\begin{minipage}{.6\textwidth}
\centering
\begin{tabular}{l|ll}
Accuracy & 0.9634 \\ \hline
Sensitivity & 0.0101 \\ \hline
Specificity & 0.9944 \\ \hline
Precision & 0.0556 \\
\end{tabular}
\subcaption{Statistics}
\end{minipage}
\end{table}

\begin{table}
\caption{Title Goes Here}
\begin{minipage}{.6\textwidth}
\centering
\begin{tabular}{l|ll}
\backslashbox{Results}{Actual} & Positive & Negative \\ \hline
Positive & 2 & 34 \\
Negative & 433 & 5850 \\
\end{tabular}
\subcaption{Confusion Matrix}
\end{minipage}
\begin{minipage}{.6\textwidth}
\centering
\begin{tabular}{l|ll}
Accuracy & 0.9261 \\ \hline
Sensitivity & 0.0046 \\ \hline
Specificity & 0.9942 \\ \hline
Precision & 0.0556 \\
\end{tabular}
\subcaption{Statistics}
\end{minipage}
\end{table}


\begin{table}
\caption{Title Goes Here}
\begin{minipage}{.6\textwidth}
\centering
\begin{tabular}{l|ll}
\backslashbox{Results}{Actual} & Positive & Negative \\ \hline
Positive & 2 & 93 \\
Negative & 102 & 6119 \\
\end{tabular}
\subcaption{Confusion Matrix}
\end{minipage}
\begin{minipage}{.6\textwidth}
\centering
\begin{tabular}{l|ll}
Accuracy & 0.9691 \\ \hline
Sensitivity & 0.0192 \\ \hline
Specificity & 0.985 \\ \hline
Precision & 0.0211 \\
\end{tabular}
\subcaption{Statistics}
\end{minipage}
\end{table}

\begin{table}
\caption{Title Goes Here}
\begin{minipage}{.6\textwidth}
\centering
\begin{tabular}{l|ll}
\backslashbox{Results}{Actual} & Positive & Negative \\ \hline
Positive & 6 & 89 \\
Negative & 334 & 5889 \\
\end{tabular}
\subcaption{Confusion Matrix}
\end{minipage}
\begin{minipage}{.6\textwidth}
\centering
\begin{tabular}{l|ll}
Accuracy & 0.933 \\ \hline
Sensitivity & 0.0176 \\ \hline
Specificity & 0.9851 \\ \hline
Precision & 0.0632 \\
\end{tabular}
\subcaption{Statistics}
\end{minipage}
\end{table}

\begin{table}
\caption{Title Goes Here}
\begin{minipage}{.6\textwidth}
\centering
\begin{tabular}{l|ll}
\backslashbox{Results}{Actual} & Positive & Negative \\ \hline
Positive & 5 & 90 \\
Negative & 194 & 6028 \\
\end{tabular}
\subcaption{Confusion Matrix}
\end{minipage}
\begin{minipage}{.6\textwidth}
\centering
\begin{tabular}{l|ll}
Accuracy & 0.955 \\ \hline
Sensitivity & 0.0251 \\ \hline
Specificity & 0.9853 \\ \hline
Precision & 0.0526 \\
\end{tabular}
\subcaption{Statistics}
\end{minipage}
\end{table}

\begin{table}
\caption{Title Goes Here}
\begin{minipage}{.6\textwidth}
\centering
\begin{tabular}{l|ll}
\backslashbox{Results}{Actual} & Positive & Negative \\ \hline
Positive & 9 & 86 \\
Negative & 426 & 5798 \\
\end{tabular}
\subcaption{Confusion Matrix}
\end{minipage}
\begin{minipage}{.6\textwidth}
\centering
\begin{tabular}{l|ll}
Accuracy & 0.919 \\ \hline
Sensitivity & 0.0207 \\ \hline
Specificity & 0.9854 \\ \hline
Precision & 0.0947 \\
\end{tabular}
\subcaption{Statistics}
\end{minipage}
\end{table}


\begin{table}
\caption{Title Goes Here}
\begin{minipage}{.6\textwidth}
\centering
\begin{tabular}{l|ll}
\backslashbox{Results}{Actual} & Positive & Negative \\ \hline
Positive & 2 & 43 \\
Negative & 102 & 6169 \\
\end{tabular}
\subcaption{Confusion Matrix}
\end{minipage}
\begin{minipage}{.6\textwidth}
\centering
\begin{tabular}{l|ll}
Accuracy & 0.977 \\ \hline
Sensitivity & 0.0192 \\ \hline
Specificity & 0.9931 \\ \hline
Precision & 0.0444 \\
\end{tabular}
\subcaption{Statistics}
\end{minipage}
\end{table}

\begin{table}
\caption{Title Goes Here}
\begin{minipage}{.6\textwidth}
\centering
\begin{tabular}{l|ll}
\backslashbox{Results}{Actual} & Positive & Negative \\ \hline
Positive & 2 & 43 \\
Negative & 338 & 5935 \\
\end{tabular}
\subcaption{Confusion Matrix}
\end{minipage}
\begin{minipage}{.6\textwidth}
\centering
\begin{tabular}{l|ll}
Accuracy & 0.9397 \\ \hline
Sensitivity & 0.0059 \\ \hline
Specificity & 0.9928 \\ \hline
Precision & 0.0444 \\
\end{tabular}
\subcaption{Statistics}
\end{minipage}
\end{table}

\begin{table}
\caption{Title Goes Here}
\begin{minipage}{.6\textwidth}
\centering
\begin{tabular}{l|ll}
\backslashbox{Results}{Actual} & Positive & Negative \\ \hline
Positive & 4 & 41 \\
Negative & 195 & 6077 \\
\end{tabular}
\subcaption{Confusion Matrix}
\end{minipage}
\begin{minipage}{.6\textwidth}
\centering
\begin{tabular}{l|ll}
Accuracy & 0.9626 \\ \hline
Sensitivity & 0.0201 \\ \hline
Specificity & 0.9933 \\ \hline
Precision & 0.0889 \\
\end{tabular}
\subcaption{Statistics}
\end{minipage}
\end{table}

\begin{table}
\caption{Title Goes Here}
\begin{minipage}{.6\textwidth}
\centering
\begin{tabular}{l|ll}
\backslashbox{Results}{Actual} & Positive & Negative \\ \hline
Positive & 4 & 41 \\
Negative & 431 & 5843 \\
\end{tabular}
\subcaption{Confusion Matrix}
\end{minipage}
\begin{minipage}{.6\textwidth}
\centering
\begin{tabular}{l|ll}
Accuracy & 0.9253 \\ \hline
Sensitivity & 0.0092 \\ \hline
Specificity & 0.993 \\ \hline
Precision & 0.0889 \\
\end{tabular}
\subcaption{Statistics}
\end{minipage}
\end{table}


\begin{table}
\caption{Intersection of Both Label Sets With Only Structural Dependencies}
\begin{minipage}{.6\textwidth}
\centering
\begin{tabular}{l|ll}
\backslashbox{Results}{Actual} & Positive & Negative \\ \hline
Positive & 59 & 1955 \\
Negative & 17 & 6448 \\
\end{tabular}
\subcaption{Confusion Matrix}
\end{minipage}
\begin{minipage}{.6\textwidth}
\centering
\begin{tabular}{l|ll}
Accuracy & 0.7674 \\ \hline
Sensitivity & 0.7763 \\ \hline
Specificity & 0.7673 \\ \hline
Precision & 0.0293 \\
\end{tabular}
\subcaption{Statistics}
\end{minipage}
\end{table}

\begin{table}
\caption{Title Goes Here}
\begin{minipage}{.6\textwidth}
\centering
\begin{tabular}{l|ll}
\backslashbox{Results}{Actual} & Positive & Negative \\ \hline
Positive & 4 & 70 \\
Negative & 336 & 5908 \\
\end{tabular}
\subcaption{Confusion Matrix}
\end{minipage}
\begin{minipage}{.6\textwidth}
\centering
\begin{tabular}{l|ll}
Accuracy & 0.9357 \\ \hline
Sensitivity & 0.0118 \\ \hline
Specificity & 0.9883 \\ \hline
Precision & 0.0541 \\
\end{tabular}
\subcaption{Statistics}
\end{minipage}
\end{table}

\begin{table}
\caption{Title Goes Here}
\begin{minipage}{.6\textwidth}
\centering
\begin{tabular}{l|ll}
\backslashbox{Results}{Actual} & Positive & Negative \\ \hline
Positive & 3 & 71 \\
Negative & 196 & 6047 \\
\end{tabular}
\subcaption{Confusion Matrix}
\end{minipage}
\begin{minipage}{.6\textwidth}
\centering
\begin{tabular}{l|ll}
Accuracy & 0.9577 \\ \hline
Sensitivity & 0.0151 \\ \hline
Specificity & 0.9884 \\ \hline
Precision & 0.0405 \\
\end{tabular}
\subcaption{Statistics}
\end{minipage}
\end{table}

\begin{table}
\caption{Union of Both Label Sets With Only Structural Dependencies}
\begin{minipage}{.6\textwidth}
\centering
\begin{tabular}{l|ll}
\backslashbox{Results}{Actual} & Positive & Negative \\ \hline
Positive & 249 & 1765 \\
Negative & 388 & 6077 \\
\end{tabular}
\subcaption{Confusion Matrix}
\end{minipage}
\begin{minipage}{.6\textwidth}
\centering
\begin{tabular}{l|ll}
Accuracy & 0.7461 \\ \hline
Sensitivity & 0.3909 \\ \hline
Specificity & 0.7749 \\ \hline
Precision & 0.1236 \\
\end{tabular}
\subcaption{Statistics}
\end{minipage}
\end{table}

